\chapter{Introduction}

Within the field of \gls{MIR}, the task of \gls{AMT} is considered to be, both an important, and challenging research problem. It describes the process of generating a symbolic notation from audio. Most instruments are melodic, where key information for transcription would be to discern pitch, onset time, and duration. This stands in contrast to percussive instruments, which are inherently event based. This sets the stage for \gls{ADT}, which is a subfield of \gls{AMT}, specifically focusing on drums and percussive instruments.~\cite{8350302}

Previously, a popular approach to \gls{ADT} was using signal processing, which later developed to using classical machine learning methods~\cite{8350302}. However in later years, deep learning has shown to be quite effective~\cite{signals4040042}.

\textcolor{red}{Provide a good introduction into the master thesis, menitoning \gls{AMT}, \gls{ADT} and why deep learning is suited for such a task.}

\section{Thesis statement}

Present the aim of the thesis here. And the \b{questions!}
How do we train a model capable of solving such a task at a high performing level. More specifically:

What architectures are suited for learning such a task?
What datasets / combination of datasets makes the model generalize best?
Of the many techniques made to help models learn this task, which ones actually help? (Ablation)

\textbf{Remember the concrete \underline{What do we want to figure out}.}