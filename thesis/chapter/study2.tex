\chapter{Dataset Study}

\section{Methodology}

This study's main goal train the selected architecture from the last study over different combinations of \gls{ADT} datasets, while zero-shot testing on the remaining ones (+ SADTP, the novel dataset introduced in this thesis) and figure out if there exists any combinations of datasets which outperforms the other on the majority, making the model generalize better. This could answer the question; "Does training on combinations of dataset lead to generalization?", and "What type of data makes our model generalize the best when it comes to \gls{ADT} related tasks?". \textcolor{red}{For this study as well; some kind of introduction to the study more or less.}

\section{Results}

\textcolor{red}{Display a table of results, micro-F1 (or maybe just micro-F1):
Non zero-shot tests are grayed out, best zero-shot test are bolded.}

\begin{center}
    \begin{tabular}{|l|cccc|}
    \hline
                     & Dataset 1 & Dataset 2 & Dataset 3 & Dataset 4 \\
    \hline
    Dataset 1        & 0.5       & 0.3          &           &           \\
    Dataset 2        & \cellcolor{blue!10}0.4       & 0.8          &           &           \\
    Dataset 1+2      & 0.8       & 0.7          &           &           \\
    Dataset 3        & \cellcolor{blue!10}\textbf{0.9}       & \cellcolor{blue!10}{0.6}          &           &           \\
    Dataset 1+2+3    & 0.95      & 0.8          &           &           \\   
    Dataset 1+4      & 0.95      & \cellcolor{blue!10}\textbf{0.75}          &           &           \\   
    Dataset 1+2+3+4  & 0.95      & 0.82          &           &           \\   
    \hline

    \end{tabular}
\end{center}

\section{Discussion}