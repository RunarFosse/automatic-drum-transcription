\chapter{Results}

\section{Architecture experiment}

Display a table of results, class-wise and micro-F1:
Best archicecture per dataset is bolded.

\begin{center}
    \begin{tabular}{|l|cccc|}
    \hline
                   & Dataset 1 & Dataset 2 & Dataset 3 & Dataset 4 \\
    \hline
    CNN            & 0.5       &           &           &           \\
    RNN            & 0.4       &           &           &           \\
    Conv-RNN       & 0.8       &           &           &           \\
    Conv-Attention & 0.9       &           &           &           \\
    Transformer    & \textbf{0.95}      &           &           &           \\   
    \hline

    \end{tabular}
\end{center}

Display the results in a barplot, to easily capture well-performing models.

Also plot enough information to be able to conclude about overall performance of models, performance on rarer instruments, etc.


\section{Dataset generalization experiment}

Display a table of results, possibly both class-wise and micro-F1 (or maybe just micro-F1):
Zero-shot tests have a grayed background, best zero-shot test are bolded.

\begin{center}
    \begin{tabular}{|l|cccc|}
    \hline
                     & Dataset 1 & Dataset 2 & Dataset 3 & Dataset 4 \\
    \hline
    Dataset 1        & 0.5       & 0.3          &           &           \\
    Dataset 2        & \cellcolor{blue!10}0.4       & 0.8          &           &           \\
    Dataset 1+2      & 0.8       & 0.7          &           &           \\
    Dataset 3        & \cellcolor{blue!10}\textbf{0.9}       & \cellcolor{blue!10}{0.6}          &           &           \\
    Dataset 1+2+3    & 0.95      & 0.8          &           &           \\   
    Dataset 1+4      & 0.95      & \cellcolor{blue!10}\textbf{0.75}          &           &           \\   
    Dataset 1+2+3+4  & 0.95      & 0.82          &           &           \\   
    \hline

    \end{tabular}
\end{center}


Display the results in a barplot, to easily capture well-performing models.

Also plot enough information to be able to conclude about overall performance of models, performance on rarer instruments, etc.


\section{Ablation experiments}

Display results and data to be able to conclude if techniques help training / give better end results.

I.e., do we converge faster? Do we converge to a better minimum? 

Could plot some loss over epochs? Need to be thorough (or average) to ensure that gains/losses are due to technique (and not hyperparameter choice, etc.). 