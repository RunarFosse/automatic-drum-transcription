\chapter{Background}

\section{Automatic Drum Transcription}

As mentioned, \gls{ADT} describes the task of transcribing symbolic notation for drums from audio. To be even more descriptive, \gls{ADT} can be split into further tasks. From least to most complex we have: \gls{DSC}, where we classify drum instruments from isolated recordings. \gls{DTD}, where we transcribe audio containing exclusively drum instruments. \gls{DTP}, where we transcribe audio containing drum instruments, and additional percussive instruments which the transcription should exclude. Finally, we have \gls{DTM}, which describes the task of drum transcription with audio containing both drum, and melodic instruments.~\cite{8350302}

In this thesis, we will focus on the most complex of these, namely \gls{DTM}. Intuitively, we want to develop a deep learning model which, given input audio, has the ability to detect and classify different drum instrument onsets (events), while selectively ignoring unrelated, melodic instruments.

This task comes with difficulties not seen in the less complex tasks. Zehren et al.\cite{signals4040042} describes one example, in where \textit{"melodic and percussive instruments can overlap and mask eachother..., or have similar sounds, thus creating confusion between instruments"}.

\textcolor{red}{
Mention some more detailed background into \gls{ADT}, \gls{DTM}, etc.
Mention some more details into how it started, how it is going, etc.
Also why this is of interest. (Maybe also what is missing)
}

\textcolor{red}{
Mention the sequence to sequence prediction.
}

\section{Related work?}

Maybe mention some \gls{ADT} related work.