\chapter{Architectures}

Finding a suitable architecture is a vital step in creating a well-performant deep learning model. By leveraging different techniques, we balance introduction of inductive biases and possibility for complexity, which hopefully can help us end up with a generalizing model.

\section{Recurrent Neural Network}

The \gls{RNN} is a standard architecture when it comes prediction on sequence data. It has been tried and tested, showing promising results for audio tasks.

The fundamental building block for \gls{RNN}s is the \textit{recurrent unit}. It iterates the whole input sequence, storing information from previous timesteps in a form of memory, through maintenance of a \textit{hidden state}. This can be extended to gaining information about future timesteps by using \textit{bidirectional} versions. In this way, prediction on current timesteps are affected by the information from surrounding timesteps. This is relevant in tasks such as \gls{ADT} as auditory information usually spreads over several timesteps, e.g. the timbre of an instrument event lingering after onset.

\begin{figure}[H]
    \centering
    \input{tikz/spectrogram}
    \caption{Visualization of adjacent timestep's influence for a bidirectional RNN.}
    \label{RNNInfluenceFigure}
\end{figure}

However, traditional \gls{RNN}s suffer from the \textit{vanishing gradient problem} due to a timestep's influence diminishing with distance, making \textit{long range dependencies} harder to learn. Different architectures have been developed to try to overcome these issues, such as the \gls{GRU} by Cho et al.~\cite{DBLP:conf/emnlp/ChoMGBBSB14}, and \gls{LSTM} by Hochreiter et al.~\cite{10.1162/neco.1997.9.8.1735}.

It has been shown that \gls{GRU}s and \gls{LSTM}s are capable of learning \gls{ADT} related tasks, and is therefore in interest to comparatively measure how their efficiency stands in regards to other architectures~\cite{Southall2016AutomaticDT, inproceedings, Vogl2017DrumTV, signals4040042}.

\subsection{Implementation}

Our \gls{RNN} architecture consists of several bidirectional recurrent units, ending in a framewise linear layer. For the \gls{BiRU}, we train both a \gls{GRU} and an \gls{LSTM} model as hyperparameters, in addition to search over number of layers $L \in \{2, 3, 4, 5, 6\}$ and hidden size $H \in \{72, 144, 288\}$, selecting the one with best performance.

\begin{figure}[H]
    \centering
    \usetikzlibrary{positioning, chains, shapes.geometric, fit, shapes, arrows.meta, calc, backgrounds}

\begin{tikzpicture}

\draw (0, 0) -- (10, 0) -- (10, 10) -- (0, 10) -- cycle;

\draw[thick,->] (5, 8) -- (5, 5) node[anchor=north] {TODO: Create RNN architecture};

\end{tikzpicture}
    \caption{RNN architecture structure.}
    \label{RNNFigure}
\end{figure}

\section{Convolutional Neural Network}

\section{Convolutional Recurrent Neural Network}

\section{Convolutional Transformer}

\section{Vision Transformer}